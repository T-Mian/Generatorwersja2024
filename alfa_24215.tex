\documentclass[12pt,twoside]{article}

%%%%---Pakiety---%%%%
\usepackage{tcolorbox}
\usepackage[top=2cm, bottom=2cm, left=2cm, right=2cm]{geometry}
\usepackage[utf8]{inputenc}
\usepackage[polish]{babel}
\usepackage[T1]{fontenc}
\usepackage{graphbox,float}
\usepackage{eso-pic}
\usepackage{graphics}
\usepackage{xcolor}
\usepackage{fontspec}
\usepackage{fancyhdr}
\usepackage{tabularx}
\usepackage{tikz}
\usepackage{datatool}

\usetikzlibrary{calc}
\pagestyle{fancy} 

    %%%%---Czcionki---%%%% 
\setmainfont[Ligatures=TeX]{Lato-Semibold} 
\definecolor{gray}{RGB}{230,230,229}
\definecolor{yellow}{RGB}{247,209,25}

\newenvironment{itemize*}%
   {\begin{itemize}%
     \setlength{\itemsep}{0pt}%
     \setlength{\parskip}{0pt}}%
   {\end{itemize}}

   \newcommand\blfootnote[1]{%
   \begingroup
   \renewcommand\thefootnote{}\footnote{#1}%
   \addtocounter{footnote}{-1}%
   \endgroup
} 

     %%%%---Nagłówek---%%%%															%%%%---Nagłówek---%%%%
\fancyhf{} 
\renewcommand{\headrulewidth}{0pt} 
\lhead { xxx\\ 
{\footnotesize Karta katalogowa/ rodziny produktu}} 
 \setlength\headheight{27pt}
\rhead{\includegraphics[width=4cm]{Foto/logo_duze_foto}} 
 
%%%%---Dokument---%%%%
\begin{document}
    %%%%---Ładowanie bazy---%%%% 
\DTLsetseparator{;} 
\DTLloaddb{myDB}{alfa_24215.csv}    % 			WPISAĆ BAZĘ CSV (BAZA POWINNA BYĆ WYGENEROWANA Z EXELA ZA POPOCĄ "ZAPISZ JAKO" I WYBRAC "PLIK CSV UTF-8 (ROZDZIELANY PRZECINKAMI)(*.CSV)") 
\DTLforeach* %						KOLEJNOŚĆ W BAZIE MUSI SIĘ ZGADZAĆ Z KOLEJNOŚCIĄ WPISANĄ  W LINIJCE 56,W PRZYPADKU DOPISANIA "x" DO NIEJ NALEŻY UŻYĆ WZORU ", \X=X," NIE ZMIENIAĆ NIC POZA "X"-EM
{myDB}% database label 
{ \kod=kod, \rodzina=rodzina, \kodbarwy=kodbarwy, \ilosc=ilosc, \psu=psu, \ppsu=ppsu, \sterowanie=sterowanie, \klasa=klasa, \kolor=kolor, \IP=IP, \IK=IK, \vin=vin, \hz=hz, \vs=vs, \zmienne=zmienne, \p=p, \flux=flux, \wyd=wyd, \pled=pled, \fluxled=fluxled, \wydled=wydled, \cri=cri, \cct=cct, \barwa=barwa, \masa=masa, \montaz=montaz, \temperatury=temperatury, \dyfuzor=dyfuzor, \wymiary=wymiary, \dopisek=dopisek, \foto=foto, \rys=rys, \rozsyl=rozsyl, \uwagi=uwagi}
{ %start for 
%%%%---Tytuł, belka i kółeczko---%%%%										%%%%---Tytuł, belka i kółeczko---%%%%
\begin{tabular}{ll}
\begin{tcolorbox}[ 	
    frame code={},
	center title,
    valign=center,
    left=0pt,
    right=0pt,
    top=0pt,
    bottom=0pt,
    colback=gray,
    width=380pt,
    height=57pt,
    enlarge left by=-4cm,
    boxsep=5pt,
    arc=28pt]
\textsc{\hspace{2cm} \Large \kod } %%                KOD OPRAWY
\end{tcolorbox}&\begin{tcolorbox}[width=57pt,
colback=yellow,
colframe=yellow,
halign=center,valign=center,
square,circular arc]
\end{tcolorbox}
\end{tabular}
\\ 

%%%%---Obrazek---%%%%												FOTO PRODUKTU
\AddToShipoutPictureBG*{
  \AtTextUpperLeft{%
    \makebox[\textwidth][r]{% Move over to right so right aligns with right of text block
      \raisebox{-10cm}{% Drop down so top aligns with top of text block                     REGULACJA OBNIŻENIA OBRAZKA WZGLĘDEM ZERA (GDZIE ZERO JEST W GÓRNEJ KRAWĘDZI STRONY)
           \includegraphics[width=0.45\textwidth,align=b]{Foto/\foto}   %              PARAMETR WIELKOŚCI ZDIĘCIA ( "=1" ZNACZY ORYGINALNY WYMIAR)
      }%
    }%
  }%
}% 

\begin{tcolorbox}[frame code={} %kreakcja podtytulu     ZASTOSOWANIE ( W ZÓŁTEJ BELCE)
    center title,
    valign=center,
    left=0pt,
    right=0pt,
    top=0pt,
    bottom=0pt,
    colback=yellow,
    width=250pt,
    height=31pt,
    enlarge left by=-4cm,
    boxsep=5pt,
    arc=15pt]\hspace{3cm} \large Zastosowanie 
\end{tcolorbox}

\begin{small} %								PUNKTOWANIE DLA ZASTOSOWANIA , JEZELI POTRZEBA DOPISAĆ/SKOPIOWAĆ  "\item " I POMIĘTAĆ O SPACJI PO SKOPIOWANIU 
\begin{itemize*} 
\item  xxx 
\item  xxx 
\item  xxx 
\item  xxx 
\end{itemize*}
\end{small}

\begin{tcolorbox}[frame code={} %kreakcja podtytulu    						CECHY PRODUKTU ( W ZÓŁTEJ BELCE)
    center title,
    valign=center,
    left=0pt,
    right=0pt,
    top=0pt,
    bottom=0pt,
    colback=yellow,
    width=250pt,
    height=31pt,
    enlarge left by=-4cm,
    boxsep=5pt,
    arc=15pt,
 	outer arc=0pt]\hspace{3cm} \large Cechy produktu
\end{tcolorbox}

\begin{small} % 									PUNKTOWANIE DLA CECH PRODUKTU ,JEZELI POTRZEBA DOPISAĆ/SKOPIOWAĆ  "\item " I POMIĘTAĆ O SPACJI PO SKOPIOWANIU
\begin{itemize*}
\item xxx 
\item  xxx 
\item  xxx 
\item  xxx 
\item  xxx  
\end{itemize*}
\end{small}

\begin{tcolorbox}[frame code={} %kreakcja podtytulu    							PARAMETRY  ( W ZÓŁTEJ BELCE)
    center title,
    valign=center,
    left=0pt,
    right=0pt,
    top=0pt,
    bottom=0pt,
    colback=yellow,
    width=250pt,
    height=31pt,
    enlarge left by=-4cm,
    boxsep=5pt,
    arc=15pt,
 	outer arc=0pt]\hspace{3cm} \large Parametry
\end{tcolorbox}
%%%																	TABELA Z PARAMETRAMI PRODUKTU		
\begin{table}[H]
\centering
\begin{scriptsize} %           ŚCIĄGA MOŻLIWYCH PARAMETRÓW;  Napięcie sterowania \vs, >\cri  ,± 0,3 kg, \temperatury,Wsp. zachowania str. świetlnego , L90B10,L80B10,Ochrona przeciprzep.-ooo°C do +ooo°C 
\begin{tabularx}{\textwidth}{lllX}
\textbf{Ogólne}                             &          				  		& \textbf{Świetlne}                                     &            			       				\\  
Kod produktu                            	& \kod        					& Moc znamionowa *                                     	& \p {} W       			       			\\ 
Kod Rodziny                            		& \rodzina      				& Strumień świetlny oprawy**                            &\flux {} lm         			       		\\ 
Kod barwy                          			& \kodbarwy       				& Wydajność oprawy **                                   & \wyd	{} lm/W         			      		\\  
Wbudowany zasilacz                          & \psu      					& CRI/Ra                                     			& >\cri          			       			\\ 
Producent zasilacza                         & \ppsu		    				& Temperatura barwowa                                   & \cct {} K           			       		\\ 
Sterowanie                           		& \sterowanie    				& Barwa światła                                     	& \barwa         			       			\\ 
Klasa ochronności                      		& \klasa {} klasa ochronności	& Kąt rozsyłu                                   		&\pled {}°     		       				\\ 
Stopień ochrony                             & \IP      				  		&                                       				&            			       				\\ 
Odporność na uderzenia                    	& \IK        				    &                                       				&            			       				\\ 
Kolor                             			& \kolor        				& \textbf{Pozostałe}                                    &            			       				\\ 
Producent źródeł światła                    & OOO         					& Waga                                     				& \masa {} ± 0.3kg         						\\  
Ilość źródeł światła                        & \ilosc        				& Żywotność                                          	& OO 000 h          			       		\\  
                             				&          				  		& Sposób montażu                                     	& \montaz       			       			\\  
                             				&          				  		& Zakres temperatur otoczenia                           & \temperatury          					\\  
\textbf{Elektryczne}                        &          				  		& \textbf{Mechaniczne}                                  &            			       				\\  
Napięcie znamionowe                         & \vin        					& Materiał obudowy                                      & ooo								        \\  
Częstotliwość sieciowa                      & \hz       					& Materiał klosza                                       & \dyfuzor        							\\  
                             				&          				  		& Wymiary (mm)                   						& \wymiary     			       			        \\  
                             				&          				  		&                                       				&            			       		\\  
\multicolumn{4}{l}{* tolerancja mocy $\pm$5\%}              \\ 
\multicolumn{4}{l}{** tolerancja strumienia świetlnego 7\% w temp. otoczenia 25\textdegree C}\\ 
\end{tabularx}
\end{scriptsize}
\end{table}
\newpage 
\begin{tcolorbox}[frame code={} %kreakcja podtytulu 		
    center title,
    valign=center,
    left=0pt,
    right=0pt,
    top=0pt,
    bottom=0pt,
    colback=yellow,
    width=250pt,
    height=31pt,
    enlarge left by=-4cm,
    boxsep=5pt,
    arc=15pt,
 	outer arc=0pt]\hspace{3cm} \large Rysunek wymiarowy
\end{tcolorbox}

\includegraphics[width=0.3\textwidth]{RysTech/\rys}

\begin{tcolorbox}[frame code={} %kreakcja podtytulu  						KRZYWA ROZSYŁU ( W ZÓŁTEJ BELCE)
    center title,
    valign=center,
    left=0pt,
    right=0pt,
    top=0pt,
    bottom=0pt,
    colback=yellow,
    width=250pt,
    height=31pt,
    enlarge left by=-4cm,
   boxsep=5pt,
    arc=15pt,
 	outer arc=0pt]\hspace{3cm} \large Krzywa rozsyłu
\end{tcolorbox}

\includegraphics[width=0.30\textwidth]{Rozsyl/\rozsyl}                      % "WYDRUK" RYSUNKU KRZYWEJ ROZSYŁU  (WIELKOŚĆ  30 PROCENT Z ORYGINAŁU)

\begin{tcolorbox}[frame code={} %kreakcja podtytulu 						MOZNTAZ / ZMIENNIE INFO , ( W ZÓŁTEJ BELCE)
    center title,
    valign=center,
    left=0pt,
    right=0pt,
    top=0pt,
    bottom=0pt,
    colback=yellow,
    width=250pt,
    height=31pt,
    enlarge left by=-4cm,
    boxsep=5pt,
    arc=15pt,
 	outer arc=0pt]\hspace{3cm} \large Akcesoria
\end{tcolorbox}

%\includegraphics[width=0.30\textwidth]{Inne/\xxx}
\includegraphics[width=0.30\textwidth]{Akcesoria/\xxx}


\blfootnote{\textbf{Oświadczenie}\\ 
Zastrzega się możliwość zmian bez uprzedzenia. Błędy i ominięcia są możliwe. Należy zawsze upewnić się czy korzystasz z najnowszej wersji dokumentu.\ \ 
\begin{flushright}
\ 
\ 3 styczeń 2023\\ 
\end{flushright}}
 
 \newpage 
 }
\end{document}